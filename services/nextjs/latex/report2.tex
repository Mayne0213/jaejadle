\documentclass[12pt,a4paper]{article}

% KoTeX 패키지
\usepackage{kotex}
\usepackage[utf8]{inputenc}

% 기타 필요한 패키지
\usepackage{graphicx}
\usepackage{geometry}
\usepackage{xcolor}
\usepackage{hyperref}
\usepackage{titlesec}
\usepackage{fancyhdr}
\usepackage{float}
\usepackage{caption}
\usepackage{subcaption}
\usepackage{enumitem}

% 페이지 설정
\geometry{
    a4paper,
    left=25mm,
    right=25mm,
    top=30mm,
    bottom=30mm
}

% 색상 정의
\definecolor{primaryblue}{HTML}{6d96c5}
\definecolor{lightblue}{HTML}{94b7d6}

% 제목 스타일 설정
\titleformat{\section}
  {\normalfont\Large\bfseries\color{primaryblue}}
  {\thesection}{1em}{}
\titleformat{\subsection}
  {\normalfont\large\bfseries\color{lightblue}}
  {\thesubsection}{1em}{}

% 헤더/푸터 설정
\pagestyle{fancy}
\fancyhf{}
\fancyhead[L]{제자들교회 웹사이트 개발 보고서}
\fancyhead[R]{주요 기능 구현 완료}
\fancyfoot[C]{\thepage}
\renewcommand{\headrulewidth}{0.4pt}

% 하이퍼링크 설정
\hypersetup{
    colorlinks=true,
    linkcolor=primaryblue,
    urlcolor=red,
    citecolor=primaryblue
}

\begin{document}

% 표지
\begin{titlepage}
    \centering
    \vspace*{2cm}

    {\Huge\bfseries 제자들교회 웹사이트 개발\\[0.5cm]}
    {\LARGE\bfseries 2차 진행 결과\\[1.5cm]}

    {\Large 주요 기능 구현 완료\\[3cm]}

    \vfill

    {\large
    프로젝트: 교회 웹사이트 구축\\[0.3cm]
    작성일: \today\\[0.3cm]
    }

\end{titlepage}

% 목차
\tableofcontents
\newpage

% 본문 시작
\section{개요}

본 보고서는 제자들교회 웹사이트 개발 프로젝트의 2차 완료 내용을 보고하기 위해 작성되었습니다.
이번 단계에서는 \textbf{공지사항}, \textbf{갤러리}, \textbf{회원가입}, \textbf{예배 영상} 기능이 완성되었으며,
교인들이 쉽게 사용할 수 있도록 직관적인 화면 구성에 중점을 두었습니다.

\subsection{완성된 기능 목록}

\begin{enumerate}[leftmargin=*]
    \item \textbf{공지사항} - 교회 소식과 행사 안내를 게시하고 관리하는 게시판
    \item \textbf{갤러리} - 교회 활동 사진을 올리고 공유하는 사진첩
    \item \textbf{회원가입} - 새로운 회원이 가입할 수 있는 양식
    \item \textbf{예배 영상} - 주일 예배와 수요 기도회 영상을 시청하는 페이지
\end{enumerate}

\subsection{주요 개선 사항}

\begin{itemize}[leftmargin=*]
    \item \textbf{속도 개선}: 공지사항 목록이 이전보다 빠르게 표시되도록 개선
    \item \textbf{화면 구성}: 모바일 화면에서도 편하게 사용할 수 있도록 설계
    \item \textbf{사용 편의성}: 클릭 한 번으로 원하는 정보에 쉽게 접근 가능
    \item \textbf{정보 저장}: 작성한 글과 사진이 안전하게 보관되도록 구성
\end{itemize}

\newpage

\section{기능별 상세 구현 내용}

\subsection{공지사항}

\subsubsection{목록 화면}

\begin{figure}[H]
    \centering
    \fbox{\includegraphics[width=0.88\textwidth]{images/announcement-list.png}}
    \caption{공지사항 목록 화면}
\end{figure}

\subsubsection{화면 구성}

공지사항 목록은 교회의 최신 소식을 한눈에 파악할 수 있도록 설계하였습니다.
각 공지사항은 카드 형태로 표시되며, 제목, 작성일, 미리보기 내용이 함께 보입니다.

\subsubsection{주요 디자인 요소}

\begin{itemize}[leftmargin=*]
    \item \textbf{화면 구성}: 각 공지사항을 흰색 카드에 담아 구분
    \item \textbf{제목 표시}: 크고 굵은 글씨로 제목을 강조
    \item \textbf{작성 정보}: 작성자 이름과 작성일을 회색 글씨로 표시
    \item \textbf{내용 미리보기}: 본문 내용의 일부를 미리 보여줌
    \item \textbf{페이지 나누기}: 한 화면에 최대 10개의 공지사항을 표시하고, 나머지는 다음 페이지에 표시
    \item \textbf{새 글 작성}: 화면 오른쪽 위에 파란색 "작성하기" 버튼 배치
\end{itemize}

\newpage

\subsubsection{글 작성 화면}

\begin{figure}[H]
    \centering
    \fbox{\includegraphics[width=0.88\textwidth]{images/announcement-write.png}}
    \caption{공지사항 작성 화면}
\end{figure}

\subsubsection{작성 화면 구성}

관리자가 새로운 공지사항을 작성할 수 있는 양식입니다.
제목과 내용을 입력하면 자동으로 작성일과 작성자가 기록됩니다.

\subsubsection{주요 기능}

\begin{itemize}[leftmargin=*]
    \item \textbf{제목 입력}: 공지사항의 제목을 입력하는 칸
    \item \textbf{내용 입력}: 본문 내용을 자유롭게 작성할 수 있는 넓은 입력 공간
    \item \textbf{글자 꾸미기}: 글자를 굵게, 기울임, 밑줄 등으로 꾸밀 수 있음
    \item \textbf{목록 만들기}: 번호 목록이나 점 목록을 쉽게 만들 수 있음
    \item \textbf{미리보기}: 작성 중인 내용을 실제 화면처럼 미리 확인 가능
    \item \textbf{저장 버튼}: 작성 완료 후 파란색 "저장" 버튼을 눌러 게시
\end{itemize}

\newpage

\subsubsection{내용 보기 화면}

\begin{figure}[H]
    \centering
    \fbox{\includegraphics[width=0.88\textwidth]{images/announcement-content.png}}
    \caption{공지사항 상세 내용 화면}
\end{figure}

\subsubsection{상세 화면 구성}

공지사항을 클릭하면 전체 내용을 볼 수 있는 화면이 나타납니다.
작성된 글의 모든 내용과 작성 정보가 깔끔하게 정리되어 표시됩니다.

\subsubsection{주요 요소}

\begin{itemize}[leftmargin=*]
    \item \textbf{제목}: 상단에 크고 굵은 글씨로 표시
    \item \textbf{작성 정보}: 작성자와 작성일을 회색 글씨로 표시
    \item \textbf{본문 내용}: 작성된 모든 내용이 읽기 편한 형태로 표시
    \item \textbf{목록으로 돌아가기}: 화면 하단의 "목록" 버튼을 눌러 목록으로 이동
    \item \textbf{수정 및 삭제}: 관리자는 "수정" 또는 "삭제" 버튼을 사용 가능
\end{itemize}

\subsubsection{기기별 화면 구성}

\begin{itemize}[leftmargin=*]
    \item \textbf{모바일}: 카드를 위아래로 나열하고, 여백을 줄여 내용 중심으로 표시
    \item \textbf{태블릿}: 적절한 여백으로 읽기 편한 너비 유지
    \item \textbf{컴퓨터}: 최대 너비를 제한하여 긴 줄의 글이 읽기 어렵지 않도록 조정
\end{itemize}

\newpage

\subsection{갤러리}

\subsubsection{사진 목록 화면}

\begin{figure}[H]
    \centering
    \fbox{\includegraphics[width=0.88\textwidth]{images/gallery-list.png}}
    \caption{갤러리 목록 화면}
\end{figure}

\subsubsection{화면 구성}

갤러리는 교회의 다양한 활동 사진을 한눈에 볼 수 있도록 구성하였습니다.
각 앨범은 대표 사진과 함께 제목이 표시되며, 클릭하면 해당 앨범의 모든 사진을 볼 수 있습니다.

\subsubsection{주요 디자인 요소}

\begin{itemize}[leftmargin=*]
    \item \textbf{격자 배치}: 사진을 바둑판처럼 가지런하게 배치
    \item \textbf{사진 크기}: 모바일은 1줄, 태블릿은 2줄, 컴퓨터는 3줄로 표시
    \item \textbf{대표 사진}: 각 앨범의 대표 이미지를 카드 위쪽에 크게 표시
    \item \textbf{제목 표시}: 사진 아래에 앨범 제목을 명확하게 표시
    \item \textbf{카드 효과}: 마우스를 올리면 카드가 살짝 위로 올라가는 효과
    \item \textbf{새 앨범}: 화면 오른쪽 위에 "새 앨범" 버튼 배치
\end{itemize}

\newpage

\subsubsection{사진 보기 화면}

\begin{figure}[H]
    \centering
    \fbox{\includegraphics[width=0.88\textwidth]{images/gallery-content.png}}
    \caption{갤러리 사진 상세 보기 화면}
\end{figure}

\subsubsection{상세 화면 구성}

앨범을 클릭하면 해당 앨범의 모든 사진을 볼 수 있습니다.
사진들은 격자 형태로 깔끔하게 정리되어 표시됩니다.

\subsubsection{주요 기능}

\begin{itemize}[leftmargin=*]
    \item \textbf{앨범 제목}: 상단에 앨범 이름을 크게 표시
    \item \textbf{사진 격자}: 모든 사진을 같은 크기의 정사각형으로 표시
    \item \textbf{사진 확대}: 사진을 클릭하면 큰 화면으로 자세히 볼 수 있음
    \item \textbf{여러 장 보기}: 모바일은 2줄, 태블릿은 3줄, 컴퓨터는 4줄로 표시
    \item \textbf{사진 추가}: 관리자는 "사진 추가" 버튼으로 새 사진 업로드 가능
    \item \textbf{목록으로}: "목록" 버튼을 눌러 갤러리 첫 화면으로 이동
\end{itemize}

\subsubsection{기기별 화면 구성}

\begin{itemize}[leftmargin=*]
    \item \textbf{모바일}: 2줄로 사진을 배치하여 작은 화면에서도 선명하게 표시
    \item \textbf{태블릿}: 3줄로 배치하여 더 많은 사진을 한 번에 표시
    \item \textbf{컴퓨터}: 4줄로 배치하여 넓은 화면을 효과적으로 활용
\end{itemize}

\newpage

\subsection{회원가입}

\subsubsection{가입 양식 화면}

\begin{figure}[H]
    \centering
    \fbox{\includegraphics[width=0.88\textwidth]{images/signUp.png}}
    \caption{회원가입 양식 화면}
\end{figure}

\subsubsection{화면 구성}

새로운 교인이나 방문자가 웹사이트에 가입할 수 있는 양식입니다.
필요한 정보를 입력하면 웹사이트의 모든 기능을 사용할 수 있습니다.

\subsubsection{입력 항목}

\begin{itemize}[leftmargin=*]
    \item \textbf{이메일 주소}: 가입할 이메일을 입력 (로그인 시 사용)
    \item \textbf{비밀번호}: 안전한 비밀번호를 설정
    \item \textbf{비밀번호 확인}: 비밀번호를 한 번 더 입력하여 확인
    \item \textbf{이름}: 본인의 이름을 입력
    \item \textbf{전화번호}: 연락 가능한 전화번호 입력
\end{itemize}

\subsubsection{주요 디자인 요소}

\begin{itemize}[leftmargin=*]
    \item \textbf{중앙 배치}: 화면 중앙에 흰색 카드 형태로 양식 배치
    \item \textbf{입력 칸}: 각 항목마다 명확한 제목과 입력 공간 제공
    \item \textbf{안내 문구}: 각 입력 칸에 어떤 내용을 입력해야 하는지 회색 글씨로 안내
    \item \textbf{가입 버튼}: 모든 정보 입력 후 파란색 "가입하기" 버튼 클릭
    \item \textbf{오류 표시}: 잘못된 정보를 입력하면 빨간색 글씨로 안내 메시지 표시
    \item \textbf{로그인 연결}: 이미 가입한 경우 "로그인" 버튼으로 이동 가능
\end{itemize}

\subsubsection{기기별 화면 구성}

\begin{itemize}[leftmargin=*]
    \item \textbf{모바일}: 화면 전체 너비를 활용하되 적절한 여백 유지
    \item \textbf{태블릿}: 중간 너비의 양식으로 읽기 편하게 표시
    \item \textbf{컴퓨터}: 최대 너비를 제한하여 중앙에 집중된 화면 구성
\end{itemize}

\newpage

\subsection{예배 영상 (Worship)}

\subsubsection{영상 목록 화면}

\begin{figure}[H]
    \centering
    \fbox{\includegraphics[width=0.88\textwidth]{images/worship-preview.png}}
    \caption{예배 영상 목록 화면}
\end{figure}

\subsubsection{화면 구성}

주일 예배와 수요 기도회 영상을 볼 수 있는 페이지입니다.
각 영상은 미리보기 이미지와 함께 제목, 날짜가 표시되어 원하는 예배를 쉽게 찾을 수 있습니다.

\subsubsection{주요 디자인 요소}

\begin{itemize}[leftmargin=*]
    \item \textbf{영상 카드}: 각 예배 영상을 카드 형태로 표시
    \item \textbf{미리보기}: 영상의 대표 이미지를 카드 위쪽에 표시
    \item \textbf{예배 제목}: 각 예배의 이름을 명확하게 표시
    \item \textbf{날짜 표시}: 예배 날짜를 회색 글씨로 표시
    \item \textbf{격자 배치}: 모바일은 1줄, 태블릿은 2줄, 컴퓨터는 3줄로 배치
    \item \textbf{재생 표시}: 각 카드에 재생 버튼 모양 표시
\end{itemize}

\newpage

\subsubsection{영상 재생 화면}

\begin{figure}[H]
    \centering
    \fbox{\includegraphics[width=0.88\textwidth]{images/worship-contents.png}}
    \caption{예배 영상 재생 화면}
\end{figure}

\subsubsection{재생 화면 구성}

영상을 클릭하면 해당 예배 영상을 시청할 수 있는 화면이 나타납니다.
큰 화면으로 영상을 감상할 수 있으며, 예배 정보도 함께 확인할 수 있습니다.

\subsubsection{주요 기능}

\begin{itemize}[leftmargin=*]
    \item \textbf{영상 재생}: 화면 상단에 큰 영상 재생 화면 배치
    \item \textbf{재생 조절}: 일시정지, 되감기, 빨리감기, 음량 조절 가능
    \item \textbf{전체 화면}: 전체 화면으로 확대하여 시청 가능
    \item \textbf{예배 제목}: 영상 아래에 예배 제목을 크게 표시
    \item \textbf{날짜 정보}: 예배 날짜를 함께 표시
    \item \textbf{목록으로}: "목록" 버튼을 눌러 영상 목록으로 이동
\end{itemize}

\subsubsection{기기별 화면 구성}

\begin{itemize}[leftmargin=*]
    \item \textbf{모바일}: 세로 화면에 맞춰 영상과 정보를 위아래로 배치
    \item \textbf{태블릿}: 적절한 비율로 영상 크기 조정
    \item \textbf{컴퓨터}: 넓은 화면을 활용하여 큰 영상 재생 화면 제공
\end{itemize}

\newpage

\section{결론}

\subsection{완료된 작업 요약}

제자들교회 웹사이트의 주요 기능 4가지가 성공적으로 완성되었습니다.
공지사항, 갤러리, 회원가입, 예배 영상 기능을 통해 교인들과 방문자들이
교회 소식을 확인하고, 사진을 공유하며, 온라인으로 예배에 참여할 수 있습니다.

\subsection{설계 원칙}

\begin{itemize}[leftmargin=*]
    \item \textbf{쉬운 사용}: 누구나 쉽게 이해하고 사용할 수 있는 직관적인 화면 구성
    \item \textbf{빠른 속도}: 공지사항을 비롯한 모든 페이지가 빠르게 표시됨
    \item \textbf{모든 기기 지원}: 휴대폰, 태블릿, 컴퓨터 모두에서 편리하게 사용 가능
    \item \textbf{안전한 보관}: 작성한 글과 사진이 안전하게 저장됨
\end{itemize}

\subsection{기대 효과}

\begin{itemize}[leftmargin=*]
    \item 교회 소식을 신속하게 전달하여 교인들의 참여도 향상
    \item 사진 공유를 통한 교회 공동체 활성화
    \item 온라인 예배 제공으로 집에서도 예배 참여 가능
    \item 새로운 방문자의 편리한 가입 절차
    \item 모바일 사용자의 편리한 접근성
\end{itemize}

\vspace{1cm}

\begin{center}
    \large
    \textbf{감사합니다.}\\[0.5cm]
    추가 피드백이나 수정 요청사항이 있으시면\\
    언제든지 말씀해 주시기 바랍니다.
\end{center}

\end{document}
