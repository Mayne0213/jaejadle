\documentclass[12pt,a4paper]{article}

% KoTeX 패키지
\usepackage{kotex}
\usepackage[utf8]{inputenc}

% 기타 필요한 패키지
\usepackage{graphicx}
\usepackage{geometry}
\usepackage{xcolor}
\usepackage{hyperref}
\usepackage{titlesec}
\usepackage{fancyhdr}
\usepackage{float}
\usepackage{caption}
\usepackage{subcaption}
\usepackage{enumitem}

% 페이지 설정
\geometry{
    a4paper,
    left=25mm,
    right=25mm,
    top=30mm,
    bottom=30mm
}

% 색상 정의
\definecolor{primaryblue}{HTML}{6d96c5}
\definecolor{lightblue}{HTML}{94b7d6}

% 제목 스타일 설정
\titleformat{\section}
  {\normalfont\Large\bfseries\color{primaryblue}}
  {\thesection}{1em}{}
\titleformat{\subsection}
  {\normalfont\large\bfseries\color{lightblue}}
  {\thesubsection}{1em}{}

% 헤더/푸터 설정
\pagestyle{fancy}
\fancyhf{}
\fancyhead[L]{제자들교회 웹사이트 개발 보고서}
\fancyhead[R]{About 섹션 구현 완료}
\fancyfoot[C]{\thepage}
\renewcommand{\headrulewidth}{0.4pt}

% 하이퍼링크 설정
\hypersetup{
    colorlinks=true,
    linkcolor=primaryblue,
    urlcolor=red,
    citecolor=primaryblue
}

\begin{document}

% 표지
\begin{titlepage}
    \centering
    \vspace*{2cm}

    {\Huge\bfseries 제자들교회 웹사이트 개발\\[0.5cm]}
    {\LARGE\bfseries 1차 진행 결과\\[1.5cm]}

    {\Large About 섹션 구현 완료\\[3cm]}

    \vfill

    {\large
    프로젝트: Next.js 기반 교회 웹사이트 구축\\[0.3cm]
    작성일: \today\\[0.3cm]
    }

\end{titlepage}

% 목차
\tableofcontents
\newpage

% 본문 시작
\section{개요}

본 보고서는 제자들교회 웹사이트 개발 프로젝트의 1차 완료 내용을 보고하기 위해 작성되었습니다.
현재까지 \textbf{About(교회 소개)} 섹션의 4개 하위 페이지가 완성되었으며,
모든 디자인은 고객님의 요구사항을 반영하여 직접 설계하고 구현하였습니다.

\subsection{완성된 페이지 목록}

\begin{enumerate}[leftmargin=*]
    \item \textbf{인사말 (Greeting)} - 담임목사 인사말 페이지
    \item \textbf{비전 (Vision)} - 교회 사훈, 사명, 심볼 소개 페이지
    \item \textbf{섬김이 (Leaders)} - 교역자 및 부서 섬김이 소개 페이지
    \item \textbf{오시는 길 (Directions)} - 교회 위치 및 교통 안내 페이지
\end{enumerate}

\subsection{사용된 기술}

\begin{itemize}[leftmargin=*]
    \item \textbf{개발 도구}: Next.js 15 (React 19)
    \item \textbf{디자인 도구}: Tailwind CSS
    \item \textbf{아이콘}: Lucide React
    \item \textbf{이미지 처리}: 자동으로 이미지 크기와 품질을 최적화
    \item \textbf{화면 대응}: 모바일 우선 설계 방식
\end{itemize}

\newpage

\section{페이지별 상세 구현 내용}

\subsection{인사말 페이지 (Greeting)}

\subsubsection{화면 캡처 (\href{https://jaejadle-church.vercel.app/greeting}{jaejadle-church.vercel.app/greeting})}

\begin{figure}[H]
    \centering
    \fbox{\includegraphics[width=0.88\textwidth]{images/greetings.png}}
    \caption{인사말 페이지 - 데스크톱 화면}
\end{figure}

\subsubsection{디자인 컨셉}

담임목사님의 따뜻한 환영 인사를 전달하는 페이지로, 간결하면서도 진중한 분위기를 연출하도록 디자인하였습니다.
좌측에는 인사말 텍스트를, 우측에는 담임목사님의 사진을 배치하여 시각적 균형을 맞추었습니다.

\subsubsection{주요 디자인 요소}

\begin{itemize}[leftmargin=*]
    \item \textbf{화면 구성}: 데스크톱에서 텍스트와 이미지를 3:2 비율로 배치
    \item \textbf{글자 크기}: 제목은 굵고 크게, 본문은 읽기 편한 중간 크기로 설정
    \item \textbf{색상}: 교회 대표 색상인 파란색 계열을 강조 포인트로 활용
    \item \textbf{이미지}: 담임목사님 사진의 모서리를 둥글게 처리하여 부드러운 인상 전달
\end{itemize}

\subsubsection{기기별 화면 구성}

\begin{itemize}[leftmargin=*]
    \item \textbf{모바일}: 텍스트와 이미지를 위아래로 배치하여 중앙 정렬
    \item \textbf{태블릿}: 글자 크기와 여백을 조정하여 최적화
    \item \textbf{데스크톱}: 텍스트와 이미지를 좌우로 배치하여 3:2 비율 유지
\end{itemize}

\newpage

\subsection{비전 페이지 (Vision)}

\subsubsection{화면 캡처 (\href{https://jaejadle-church.vercel.app/vision}{jaejadle-church.vercel.app/vision})}

\begin{figure}[H]
    \centering
    \fbox{\includegraphics[width=0.85\textwidth]{images/vision1.png}}
    \caption{비전 페이지 - 사훈 섹션}
    \vspace{-0.3cm}
\end{figure}

\begin{figure}[H]
    \centering
    \fbox{\includegraphics[width=0.85\textwidth]{images/vision2.png}}
    \caption{비전 페이지 - 교회 사명 섹션}
    \vspace{-0.3cm}
\end{figure}

\begin{figure}[H]
    \centering
    \fbox{\includegraphics[width=0.85\textwidth]{images/vision3.png}}
    \caption{비전 페이지 - 교회 심볼 섹션}
\end{figure}

\subsubsection{디자인 컨셉}

교회의 핵심 가치와 정체성을 시각적으로 표현하는 가장 중요한 페이지입니다.
사훈, 사명, 심볼 3개의 섹션으로 구성하여 교회의 비전을 명확하게 전달하도록 설계하였습니다.

\subsubsection{섹션별 구현 내용}

\paragraph{1. 사훈 섹션}

\begin{itemize}[leftmargin=*]
    \item \textbf{화면 구성}: 3개 항목을 가로로 균등하게 배치
    \item \textbf{아이콘}: 각 사훈을 상징하는 아이콘을 회색 원 안에 배치
    \item \textbf{번호 표시}: 왼쪽 위에 파란색 원형 배지로 번호를 표시
    \item \textbf{구분선}: 항목 사이에 세로 선을 넣어 구분 (태블릿 이상 화면)
    \item \textbf{항목}:
        \begin{enumerate}
            \item 인화단결 - 하나로 이루는 공동체
            \item 연구개척 - 새로운 가치 창조
            \item 신속정확 - 정확한 판단과 실행
        \end{enumerate}
\end{itemize}

\paragraph{2. 교회 사명 섹션}

\begin{itemize}[leftmargin=*]
    \item \textbf{중앙 요소}: 데스크톱 화면 중앙에 45도 기울인 다이아몬드 모양 "FAITH" 로고 배치
    \item \textbf{색상 효과}: 파란색에서 밝은 파란색으로 자연스럽게 변하는 그라데이션 적용
    \item \textbf{항목 배치}: 6개 항목을 좌우 3개씩 대칭으로 배치
    \item \textbf{카드 모양}: 흰색 배경에 회색 테두리, 마우스를 올리면 그림자 효과 표시
    \item \textbf{색상 체계}: 각 항목마다 고유한 파란색 계열로 구분
    \item \textbf{배치 방향}: 왼쪽 항목은 왼쪽 정렬, 오른쪽 항목은 오른쪽 정렬하여 중앙 집중
\end{itemize}

\paragraph{3. 교회 심볼 섹션}

\begin{itemize}[leftmargin=*]
    \item \textbf{배경 색상}: 전체 섹션에 밝은 파란색 배경 적용
    \item \textbf{화면 구성}: 왼쪽에 로고, 오른쪽에 4개의 설명 카드 배치
    \item \textbf{로고 영역}: 흰색 정사각형 카드에 그림자 효과 적용
    \item \textbf{설명 카드}: 각 심볼 요소(나무, 하트, 사람, 파란색)를 개별 카드로 설명
    \item \textbf{강조 텍스트}: 각 설명의 주요 단어를 파란색 굵은 글씨로 강조
\end{itemize}

\subsubsection{기기별 화면 구성}

\begin{itemize}[leftmargin=*]
    \item \textbf{모바일}:
        \begin{itemize}
            \item 사훈 3개 항목을 위아래로 배치
            \item 교회 사명의 중앙 다이아몬드 로고는 숨김
            \item 6개 사명 항목을 위아래로 배치
        \end{itemize}
    \item \textbf{태블릿}:
        \begin{itemize}
            \item 사훈 3개 항목을 좌우로 배치
            \item 교회 사명을 2줄로 배치
        \end{itemize}
    \item \textbf{데스크톱}:
        \begin{itemize}
            \item 중앙 다이아몬드 FAITH 로고 표시
            \item 좌우 대칭 배치 완성
        \end{itemize}
\end{itemize}

\newpage

\subsection{섬김이 페이지 (Leaders)}

\subsubsection{화면 캡처 (\href{https://jaejadle-church.vercel.app/leaders}{jaejadle-church.vercel.app/leaders})}

\begin{figure}[H]
    \centering
    \fbox{\includegraphics[width=0.88\textwidth]{images/leaders.png}}
    \caption{섬김이 페이지 - 데스크톱 화면 (카드 기반 레이아웃)}
\end{figure}

\subsubsection{디자인 컨셉}

교역자와 부서 섬김이를 소개하는 페이지로, 깔끔하고 현대적인 카드 기반 디자인을 채택하였습니다.
정사각형 카드 레이아웃으로 시각적 통일성을 강조하며, 각 교역자의 프로필 사진과 정보를 명확하게 전달합니다.

\subsubsection{주요 디자인 요소}

\paragraph{1. 교역자 섹션}

\begin{itemize}[leftmargin=*]
    \item \textbf{화면 구성}: 3줄로 배치 (모바일 1줄, 태블릿 2줄, 데스크톱 3줄)
    \item \textbf{카드 모양}:
        \begin{itemize}
            \item 정사각형 비율로 통일감 부여
            \item 흰색 배경에 얇은 회색 테두리
            \item 미세한 그림자 효과로 입체감 표현
        \end{itemize}
    \item \textbf{프로필 구성}:
        \begin{itemize}
            \item 상단: 원형 프로필 사진 (고정 크기)
            \item 중앙: 이름 (크고 굵은 글씨) + 하단 구분선
            \item 하단: 관심 분야 + 이메일 정보
        \end{itemize}
    \item \textbf{프로필 사진}:
        \begin{itemize}
            \item 원형으로 잘라 부드러운 인상
            \item 이미지 크기와 품질을 자동 최적화
            \item 사진 비율을 유지하면서 영역에 맞게 자동 조정
            \item 화면 크기가 변해도 사진이 찌그러지지 않도록 처리
        \end{itemize}
    \item \textbf{표시 정보}:
        \begin{itemize}
            \item 담임목사: 김경일 목사
            \item 부목사: 김택 부목사
            \item 전도사: 유경민, 황성진 전도사
        \end{itemize}
    \item \textbf{추가 정보}:
        \begin{itemize}
            \item 학위 및 졸업 연도
            \item 관심 분야 (쉼표로 구분)
            \item 이메일 주소
        \end{itemize}
\end{itemize}

\subsubsection{기기별 화면 구성}

\begin{itemize}[leftmargin=*]
    \item \textbf{모바일}: 1줄로 배치, 정사각형 비율로 균형잡힌 화면
    \item \textbf{태블릿}: 2줄로 배치, 카드 간격 최적화
    \item \textbf{데스크톱}: 3줄로 배치, 최대 너비를 제한하여 가독성 확보
\end{itemize}

\newpage

\subsection{오시는 길 페이지 (Directions)}

\subsubsection{화면 캡처 (\href{https://jaejadle-church.vercel.app/directions}{jaejadle-church.vercel.app/directions})}

\begin{figure}[H]
    \centering
    \fbox{\includegraphics[width=0.88\textwidth]{images/direction.png}}
    \caption{오시는 길 페이지 - 데스크톱 화면}
\end{figure}

\subsubsection{디자인 컨셉}

교회를 방문하고자 하는 분들을 위한 실용적인 안내 페이지입니다.
지도와 함께 버스, 지하철 등 다양한 교통수단별 상세 안내를 제공하도록 설계하였습니다.

\subsubsection{주요 디자인 요소}

\paragraph{1. 지도 영역}

\begin{itemize}[leftmargin=*]
    \item \textbf{크기}: 모바일은 작게, 태블릿은 중간, 데스크톱은 크게 표시
    \item \textbf{디자인}: 모서리를 둥글게 처리한 회색 영역
    \item \textbf{향후 계획}: 카카오맵 또는 네이버맵 연동 예정
\end{itemize}

\paragraph{2. 교회 정보 영역}

\begin{itemize}[leftmargin=*]
    \item \textbf{화면 구성}: 왼쪽에 교회명, 오른쪽에 주소와 연락처
    \item \textbf{아이콘}: 위치 아이콘과 전화 아이콘을 파란색 원 안에 배치
    \item \textbf{구분선}: 하단에 회색 선으로 섹션 구분
    \item \textbf{기기별 배치}: 모바일은 위아래로, 데스크톱은 좌우로 배치
\end{itemize}

\paragraph{3. 교통 안내 영역}

\begin{itemize}[leftmargin=*]
    \item \textbf{버스 이용 시}:
        \begin{itemize}
            \item 왼쪽: 버스 아이콘 + "버스 이용 시" 제목
            \item 오른쪽: 상세 안내 (일반버스, 광역버스 구분)
            \item 세로 선으로 아이콘과 내용 구분
        \end{itemize}
    \item \textbf{지하철 이용 시}:
        \begin{itemize}
            \item 동일한 배치 방식 적용
            \item 지하철 아이콘 사용
            \item KTX 연계 정보 추가 제공
        \end{itemize}
    \item \textbf{강조 표시}: 광역버스, KTX 연계는 파란색 굵은 글씨로 강조
\end{itemize}

\subsubsection{기기별 화면 구성}

\begin{itemize}[leftmargin=*]
    \item \textbf{모바일}:
        \begin{itemize}
            \item 아이콘과 텍스트를 좌우로 배치
            \item 교회 정보를 위아래로 정렬
        \end{itemize}
    \item \textbf{태블릿}:
        \begin{itemize}
            \item 아이콘과 텍스트를 위아래로 배치
            \item 교회 정보를 좌우로 배치
        \end{itemize}
    \item \textbf{데스크톱}:
        \begin{itemize}
            \item 모든 요소를 좌우로 배치
            \item 아이콘 크기 확대
        \end{itemize}
\end{itemize}

\newpage

\section{결론}

\subsection{완료된 작업 요약}

제자들교회 웹사이트의 About 섹션 4개 페이지가 성공적으로 완성되었습니다.
모든 디자인은 교회의 브랜드 아이덴티티를 반영하여 직접 설계하였으며,
모바일, 태블릿, 데스크톱 3가지 화면 크기에서 최적화된 사용자 경험을 제공합니다.

\subsection{디자인 원칙}

\begin{itemize}[leftmargin=*]
    \item \textbf{간결함}: 불필요한 요소를 제거하고 핵심 정보에 집중
    \item \textbf{일관성}: 모든 페이지에 통일된 디자인 적용
    \item \textbf{접근성}: 누구나 쉽게 사용할 수 있는 직관적인 화면 구성
    \item \textbf{반응성}: 모든 기기에서 최적의 사용 경험 제공
\end{itemize}

\subsection{기대 효과}

\begin{itemize}[leftmargin=*]
    \item 교회 정체성과 비전의 명확한 전달
    \item 방문자에게 친근하고 전문적인 첫인상 제공
    \item 모바일 사용자의 편리한 정보 접근
    \item 교회 브랜드 이미지 강화
\end{itemize}

\vspace{1cm}

\begin{center}
    \large
    \textbf{감사합니다.}\\[0.5cm]
    추가 피드백이나 수정 요청사항이 있으시면\\
    언제든지 말씀해 주시기 바랍니다.
\end{center}

\end{document}
